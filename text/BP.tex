\documentclass[a4paper,12pt]{article}
\usepackage{a4wide}
\usepackage[T1]{fontenc}
\usepackage{lmodern}
\usepackage[utf8]{inputenc}
\usepackage{xcolor}
\usepackage[czech,english]{babel}
\usepackage[pdftex, final]{graphicx}
% \usepackage[pdftex, final, colorlinks=true]{hyperref}
\usepackage{verbatim}
\usepackage{alltt}
\usepackage{paralist}
\usepackage{mdwlist}
\usepackage{subfig}
\usepackage[final]{pdfpages}
%\usepackage[hyphens]{url}
%\PassOptionsToPackage{hyphens}{url}

\usepackage[final,pdftex,colorlinks=false,breaklinks=true]{hyperref}
\usepackage[hyphenbreaks]{breakurl}

%%%%%%%%%%%%%%%%%%%%%%%%%
% pro podmineny preklad
% false je defaultně


% \newif\ifbc % Pouze do bakalářské práce
%  \bctrue

%%%%%%%%%% fancy %%%%%%%%%%%
\usepackage{fancyhdr}

\fancyhead[L]{ČVUT v Praze}

\setlength{\headheight}{16pt}

% \usepackage{stdpage}


%%%%%%%%%%%% rozmery %%%%%%%%%%%%%%%%%%
\usepackage[%
%top=40mm,
%bottom=35mm,
%left=40mm,
%right=30mm
top=40mm,
bottom=35mm,
left=35mm,
right=25mm
]{geometry}


\renewcommand\baselinestretch{1.3}
\parskip=0.8ex plus 0.4ex minus 0.1 ex

\newcommand{\klicslova}[2]{\noindent\textbf{#1: }#2}
\newcommand{\modul}[1]{\emph{#1}}
\author{Štěpán Turek}
% \pagecolor{darkGrey}
\newcommand{\necislovana}[1]{%
\phantomsection
\addcontentsline{toc}{section}{#1}
\section*{#1}
\markboth{\uppercase{#1}}{}
}

%%%%%%%%%%%%%%%%%%%%%%%%%%%%%%
\begin{document}
\pagestyle{empty}

\input{titulbp}
\newpage
\input{listsezadanim} % resi si zalomeni sam





\begin{abstract}

\bigskip

\klicslova{Klíčová slova}{GIS, GRASS, SAGA}

\end{abstract}

\selectlanguage{english}
\begin{abstract}

\bigskip

\klicslova{Keywords}{GIS, GRASS, SAGA}

\end{abstract}
\selectlanguage{czech}


\newpage
\newcommand{\odsaditodzhora}{\hskip1pt\vfill}

\odsaditodzhora
\noindent Prohlášení

Prohlašuji, že bakalářskou práci na téma „Implementace podpory WMS 
do programů GRASS GIS a SAGA GIS“ jsem vypracoval samostatně. Pou\-žitou
literaturu a podkladové materiály uvádím v seznamu
zdrojů.


\begin{flushleft}
\begin{tabular}{cp{0.3\textwidth}c}
V Praze dne .................
& 
&
..................................
\\
&&
(podpis autora)
\end{tabular}

\end{flushleft}
\newpage

\odsaditodzhora
\noindent Poděkování

Chtěl bych zde poděkovat vedoucímu mé bakalářské práce, Ing. Martinu Landovi, za 
cenné rady a čas, který mi věnoval při konzultacích.  Velké díky také patří mým
rodičům za jejich podporu při studiu.
\newpage

\newpage
\tableofcontents


\newpage
\pagestyle{fancy}

\necislovana{Úvod}



GRASS GIS (\url{http://grass.osgeo.org}) je geografický informační systém, šířený  pod svobodnou licencí GNU GPL.  
Historie GRASS GIS sahá do roku 1982. V tomto roce americká armáda začala vyvíjet software, který by ji pomohl se správou rozsáhlých oblastí z hlediska ochrany 
životního prostředí. Příčinou byla nová, přísnější legislativa ve vztahu k tomuto odvětví, která kladla na majitelé pozemků větší požadavky a amarická armáda patří mezi jedny z největších vlastníku půdy v USA.
 
Významným milníkem pro GRASS je rok 1995, kdy se americká armáda z toho projektu stáhla a kolem GRASS GIS se začala rodit komunita dobrovolníků. 
Dnes je komunita vývojářů rozprostřena po celém světě, z nichž se většina rekrutuje z univerzit a výzkumných ústavů. 

GRASS je jeden z nejrobustnějších svobodných GIS software. Umožnuje pracovat s vektorovými a rastrovými daty.
Jelikož se GRASS vyvíjí již 25 let, jedním z dědictví takto dlouhého vývoje je, že  jádro je napsáno procedurálně v jazyce C. 
V poslední době je snaha vývojářů učinit GRASS použitelnější pro méně pokročilé uživatele. Z tohoto důvodu bylo vyvinuto nové GUI, které se intenzivně rozvíji. 
Funkcionality GRASS GIS jsou do programu implementovány v podobě modulů. Moduly jsou do programu integrovány pomocí API, které existuje v C a Python verzi.

SAGA GIS (\url{http://www.saga-gis.org/}) je menší open source projekt, jehož vývoj začal na
 univerzitě v Goettingenu kolem roku 2004. Nyní je projekt šířen pod  licencí GNU GPL. Vývojařská 
komunita je mezinárodní s těžištěm na domovské universitě. Hlavní síla SAGA GIS spočívá v práci s rastrovými daty. Program je schopen pracovat i s daty vektorovými. 
SAGA GIS je napsán v jazyce  C++.  Koncept programu je rovněž modulární. Moduly jsou do jádra integrovány pomocí API v jazyce C++.

Web Map Service (WMS) je standard\footnote{\url{http://www.opengeospatial.org/standards/wms}}, který definuje rozhraní mezi klientem 
a serverem pro získávání georeferencovaných dat v rastrových formátech \footnote{ vektorová data ve formátu SVG nebo CGM.} (např. JPEG, TIFF, PNG). 
Jde o otevřený standard, který je vyvíjen organizací Open Geospatial Consortium\footnote{\url{http://www.opengeospatial.org/}}.  


\newpage
\section{Stručný úvod do WMS}

Komunikace mezi klientem a serverem probíhá pomocí protokolu HTTP, kdy klient zašle serveru požadavek, 
server požadavek zpracuje a  zašle klientu soubor s odpovědí, což může být rastr v požadovaném formátu nebo soubor s metadaty.

WMS standard je definován v několika verzích.

\begin{table}[h]
\centering
\begin{tabular}{|c|c|c|c|}      \hline
  Číslo verze  & Rok vydání  \\ \hline
  1.0.0        &  2000       \\ \hline
  1.1.0        &  2001       \\ \hline
  1.1.1        &  2002       \\ \hline
  1.3.0        &  2004       \\ \hline
\end{tabular}
\caption{Verze WMS}
\label{tab:verze}
\end{table}

V dnešní době téměř všechny severy podporují verze 1.3.0 a 1.1.1. Protože
změny ve verzi 1.3.0 nejsou uvedeny ve standardu, jsou popsány v příloze [TODO odkaz].


\subsection{Komunikace klient-server}


Požadavak klienta na WMS server může být zaslán pomocí dotazovacích metod GET a POST protokolu HTTP. Pomocí těchto metod je klient schopen
serveru předat parametry, na základě kterých server vytvoří odpověď. WMS standard vyžaduje podporu metody GET, zatímco podpora metody POST je volitelná. 

\subsubsection{HTTP GET}

Metoda GET předává parametry jako součást URL. URL adresa  je řetězec znaků, který reprezentuje adresu zdroje informací. Tento řetězec má pevně danou strukturu:
	
\begin{alltt}\footnotesize
	protokol://SERVER: port / cesta k dokumentu ? parametry
\end{alltt}
	
čemuž odpovídá tento příklad WMS dotazu metodou GET:

\begin{alltt}\footnotesize
\url{http://wms.cuzk.cz:80/wms.asp?REQUEST=GetCapabilities&VERSION=1.1.1}
\end{alltt}

\newpage

Protože port číslo 80 je implicitní pro protokol HTTP, není třeba ho zadávat.   
Aby server byl schopen zpracovat část s parametry, jsou určeny znaky (viz. tab \ref{tab:myfirsttable}) se speciálními funkcemi. 

\begin{table}[h]
\centering
\begin{tabular}{|c|l|}      \hline
  Znak      &    Funkce				\\ \hline
   ?        &  Začátek řetězce parametrů.      	\\ \hline
   \&       &  Oddělovač parametrů.   		\\ \hline
   =        &  Oddělovač názvu parametru a jeho hodnoty.    \\ \hline
   ,        &  Oddělovač jednotlivých položek, pokud parametr obsahuje více hodnot\\ \hline
   +        &  Reprezentuje mezeru. 	\\ \hline
\end{tabular}
\caption{Znaky v URL se speciální funkcí}
\label{tab:myfirsttable}
\end{table}

Pokud je potřeba v URL uvést tyto vyhrazené znaky, lze použít URL kódování\footnote{\url{hhttp://www.w3schools.com/tags/ref_urlencode.asp}}.

\subsubsection{HTTP POST}

Metoda POST neposílá parametry v URL adrese ale přenáší je v těle POST zprávy.


\subsection{Komunikace server-klient}

Odpovědí serveru na WMS dotaz je soubor, který se odesílá protokolem Multipurpose Internet Mail Extensions\footnote{\url{http://mgrand.home.mindspring.com/mime.html}}. Tento protokol umožnuje zasílat soubory pomocí protokolu HTTP.


\subsection{Fungování WMS v praxi}

 
Nejčastěji uživatel získává data z WMS serveru pomocí klienta, který je součástí GIS nebo jiného programu. Každý klient na pozadí vytváří WMS dotazy a jejich 
tvorba je v této kapitole popsána na reálných příkladech. 


\newpage
Jediné, co je potřeba znát pro zahájení komunikace s WMS serverem, je jeho URL. V tomto případě:
\begin{alltt}\footnotesize
\url{http://geoportal.cuzk.cz/WMS_ZABAGED_PUB/WMService.aspx}
\end{alltt}

Nyní musí klient vytvořit dotaz typu GetCapabilities, aby zjistil informace o datech, která server poskytuje a o parametrech pro ostatní WMS dotazy (GetMap a GetFeatureInfo).
Toto učiní přidáním parametrů k URL adrese WMS serveru:

\newcommand{\CUZKgetCap}{http://geoportal.cuzk.cz/WMS_ZABAGED_PUB/WMService.aspx?SERVICE=WMS&REQUEST=GetCapabilities&VERSION=1.3.0}
\begin{alltt}\footnotesize
\href{\CUZKgetCap}{http://geoportal.cuzk.cz/WMS_ZABAGED_PUB/WMService.aspx?}
\href{\CUZKgetCap}{SERVICE=WMS&REQUEST=GetCapabilities&VERSION=1.3.0}
\end{alltt}

 Dotaz typu GetCapabilities obsahuje parametry, které jsou společne pro všechny typy, protože definují způsob komunikace.
\begin{itemize}
  \item Parametr SERVICE sděluje serveru že se jedná o WMS dotaz. 
  \item Parametr REQUEST popisuje typ dotazu. 
  \item Parametr VERSION popisuje jaká verze WMS standardu bude použita.
\end{itemize}

Aby byl server schopen správně zpracovat WMS dotazy, musí se klient a server dohodnout na verzi WMS, ve které bude probíhat následná komunikace.
Toto je součástí dotazu typu GetCapabilities. Pokud není v dotazu GetCapabilities uveden parametr VERSION, server odpoví ve formátu nejvyšší podporované verze. 
Pokud klient explicitně požaduje určitou verzi, server odpoví v dané verzi, pokud ji podporuje. Jak bylo výše zmíněno, dnes naprostá většina serverů podporuje 
verze WMS standardu 1.1.1 a 1.3.0. Pokud klient podporuje tyto dvě verze, problém s nekompatibilitou v podstatě odpadá.

Na tento dotaz server vrátí soubor ve formátu XML. 

Informace o verzi je uvedena jako atribut úvodního kořenového elementu:
\begin{alltt}\footnotesize
<WMS_Capabilities...    ...version="1.3.0">
\end{alltt}

Kořenový element obsahuje elementy <Service> a <Capability>.

Element  <Service> obsahuje informace o WMS Serveru a poskytovateli dat.
Druhý element <Capability> je mnohem důležitější, protože poskytuje všechny informace, které jsou potřeba pro další komunikaci s WMS serverem.

\begin{alltt}\footnotesize
<Capability>
    <Request>
          ...
       <GetMap>
            <Format>image/png</Format>
            <Format>image/jpeg</Format>
          ...
       </GetMap>	
       <GetFeatureInfo>
            <Format>text/html</Format>
            <Format>text/xml</Format>
              ...
       </GetFeatureInfo>
           ....
\end{alltt}
V této částí jsou uvedeny formáty odpovědí ve tvaru protokolu MIME pro dotaz typu GetMap a GetFeatureInfo. V tomto případě WMS server poskytuje  mapy jako rastry ve formátu PNG, JPEG a na 
dotaz typu GetFeatureInfo může vrátit odpověď ve formátu html nebo xml.  

Velmi důležitý je element <Layer>, který obsahuje informace o mapové vrstvě. Všechny vrstvy jsou uspořádány do stromové struktury s jedním kořenovým elementem.

\begin{alltt}\footnotesize
<Capability>
    ...
  <Layer>
   <Title>ZABAGED</Title>
     <CRS>EPSG:3035</CRS>
     <CRS>EPSG:3034</CRS>
     <CRS>EPSG:4326</CRS>
      ...
     <BoundingBox CRS="EPSG:3035" minx="4434628.0972282" miny="2778319.58676976"
                                  maxx="4987359.29769667" maxy="3190250.19895492"/>
     <BoundingBox CRS="EPSG:3034" minx="4109720.95957183" miny="2382975.60863023"
                                  maxx="4643932.77142764" maxy="2780500.92255097"/>
      ...
\end{alltt}

Zde jsou uvedeny název kořenové vrstvy <Title> a projekce <CRS> v nichž je dostupná. Název vrstvy se uvádí pomocí elementů <Title> a <Name>. Element
<Title> je název vrstvy ve formátu pro člověka pochopitelném a má pouze informativní charakter, zatímco element <Name> slouží jako unikátní klíč, 
pod kterým je možné danou vrstvu jednoznačně identifikovat v rámci WMS serveru. Jelikož kořenová vrstva nemá element <Name>, není možno poslat požadavek na data této vrstvy. Vrstvy, které neposkytují 
žádná data, se uvádí z důvodu dědičnosti. 

Jelikož projekce je atribut, který je v rámci stromu vrstev děděn a projekce v příkladu jsou uvedeny v kořenovém elementu <Layer>, jsou všechny vrstvy tohoto serveru dostupné v těchto projekcích.
Jakákoliv vrstva v tomto stromu může mít definovány další projekce, které budou děděny všemi jejími následovníky ve stromu. 

Element <BoundingBox> reprezentuje obdélník definovaný minimálními a maximálními souřadnicemi v systému projekce uvedené v atributu CRS, který vymezuje rozsah poskytovaných dat. 
Tento element se také ve stromu dědí, pokud je však v potomcích vrstvy nově defininován pro stejnou projekci, nahrazuje děděný element.

 Právě dědičnost je důvodem, proč jsou vrstvy uspořádány do stromové struktury. Díky této struktuře není potřeba v tomto případě u každé vrstvy
uvádět všech 16 souřadnicových systému a obdélníků, čímž dochází k úspoře dat, která jsou přenášena mezi serverem a klientem a také k větší přehlednosti a stručnosti XML souboru.

\newpage

\begin{alltt}\footnotesize
<Capability>
  <Layer>
    <Title>Kořenová vrstva bez dat, chybí name></Title>
    <Layer>
      <Layer>
        <Title>Vrstva č. 1</Title>
        <Name>vrstva1</Name>
      </Layer>
    <Layer>
    <Layer>
      <Layer>
        <Title>Vrstva č. 2</Title>
        <Name>vrstva2</Name>
        <Layer>
          <Title>Vrstva č. 3</Title>
          <Name>vrstva3</Name>
        </Layer>
        <Layer>
          <Title>Vrstva č. 3</Title>
          <Name>vrstva4</Name>
        </Layer>
      </Layer>
    </Layer>
</Capability>
\end{alltt}



Tato delší ukázka ilustruje uspořádání vrstev ve stromové struktuře. Kořenová vrstva se větví v první úrovní  na vrstvy vrstva1, 
vrstva2. Vrstvy vrstva1, vrstva3 a vrstva4 jsou tzv. listy stromu. Takto se nazývají elememnty ve stromové struktuře, které nemají 
žádné potomky. 



\begin{alltt}\footnotesize

<Layer queryable="0" opaque="false" noSubsets="0">
    <Name>GR_CR4</Name>
    <Title>MČR 1 : 1 000 000</Title>
    <Style>
        <Name>Default</Name>
        <Title>Default</Title>
          ...
    </Style>
</Layer>
\end{alltt}


Každá vrsta reprezentuje určitá data, která však mohou být zobrazená rozličnými způsoby. Zbůsob zobrazení definuje tag style. V tomto případě  je pro vrstvu MČR 1 : 1 000 000 
dostupný pouze jeden styl. Vztah elementů <Name> a <Title> pro <Style> je stejný jako v elementu <Layer>.

Díky dotazu GetCapabilities máme všechny potřebné informace k získání požadovaných dat z WMS serveru. 
Toto provedeme pomocí dotazu typu GetMap, v tomto tvaru:



\newcommand{\CUZKgetMap}{http://geoportal.cuzk.cz/WMS_ZABAGED_PUB/WMService.aspx?SERVICE=WMS&REQUEST=GetMap&VERSION=1.3.0&LAYERS=GR_CR4&STYLES=Default&FORMAT=image/png&CRS=EPSG:4326&BBOX=48.093144621684,11.6163532829661,51.4980192528993,19.0628256634265&WIDTH=800&HEIGHT=600}
\begin{alltt}\footnotesize
\href{\CUZKgetMap}{http://geoportal.cuzk.cz/WMS\_ZABAGED\_PUB/WMService.aspx?}
\href{\CUZKgetMap}{SERVICE=WMS\&REQUEST=GetMap\&VERSION=1.3.0\&}
\href{\CUZKgetMap}{LAYERS=GR\_CR4\&STYLES=Default\&FORMAT=image/png\&CRS=EPSG:4326\&}
\href{\CUZKgetMap}{BBOX=48.093144621684,11.6163532829661,51.4980192528993,19.0628256634265\&}
\href{\CUZKgetMap}{WIDTH=800\&HEIGHT=600}
\end{alltt}


\begin{itemize}
  \item Parametr FORMAT definuje formát v němž bude vygenerována odpověď s mapou. 
  \item Parametr LAYERS obsahuje požadované vrstvy. Pořadí v jakém budou vrstvy ve výsledné mapě zobrazeny se uvádí v pořadí v jakém jsou uvedeny. Názvy jednotlivých vrstev 
        jsou odděleny čárkou. Vrstva která je ve uvedena vlevo od dalších vrstev je zobrazena nad těmito vrstvami.
  \item Parametru STYLES obsahuje styly vybraných vrstev. Styly se uvádí ve stejném pořadí jako vrstvy.    
  \item Paramentr CRS definuje projekci výsledné mapy. 
  \item Parametr BBOX určuje v jednotkách požadované projekce obdélník, ve kterém požadujeme data. Hodnoty mohou být i vně obdélníku definovaném v GetCapabilites,
	kde pouze informuje o rozsahu poskytovaných dat.  
  \item Parametry WIDTH a HEIGHT definuje počet pixelů výsledného obrázku. 
\end{itemize}

Všechny tyto parametry byly zvoleny na základě informací, které jsme získali pomocí dotazu GetCapabilities. 

Na základě tohoto dotazu obdržíme jako odpověď rastr ve formátu PNG s přehledovou mapou České republiky:

 \includegraphics[scale=0.5]{figures/GetMapResponse}


Poslední typ dotazu je volitelný a nazývá se GetFeatureInfo. Ten dotaz, narozdíl od předchozích, nemusí server podporovat.
Dotaz slouží k získán informací o prvcích ve vrstvě. Pokud je v elementu <Layer> uveden argument  queryable 
s hodnotou 1, je možné na tuto vrstvu aplikovat dotaz GetFeatureInfo.  

Jako příklad pro tento typ dotazu je použit WMS server Středočeského kraje. Jedna z WMS služeb, které tento server poskytuje, jsou zóny integrovaného dopravního systému dostupné
na :\url{http://mapy.kr-stredocesky.cz/ids_zony_wms}. 
Nejprve se dotazem GetaCapabilities zjistí informace o WMS serveru a jím poskytovaných datech:

\newcommand{\StredoceskygetCap}{http://mapy.kr-stredocesky.cz/ids_zony_wms?SERVICE=WMS&REQUEST=GetCapabilities}
\begin{alltt}\footnotesize
\href{\StredoceskygetCap}{http://mapy.kr-stredocesky.cz/ids\_zony\_wms?}
\href{\StredoceskygetCap}{SERVICE=WMS\&REQUEST=GetCapabilities\&VERSION=1.3.0}
\end{alltt}

Z odpovědi je patrné, že server obsahuje pouze jednu vrstvu Zony SID, se kterou lze dále pracovat, protože její rodičovská vrstva nemá atribut <Name> nutný pro dotazy GetMap a GetFeatureInfo.
Vrstva je dostupná v jediném souřadnicovém systému. Za povšimnutí stojí opakované uvedení souřadnicové systému a stejného obdélníku BoundingBox v němž jsou poskytována data 
jak v kmenové vrstvě, tak v jejím potomku Zony SID. Jelikož se tyto elementy vrstvy dědí, stačilo by je uvést pouze v kořenové vrstvě.

Na základě informací z dotazu GetCapabilities byl vytvořen dotaz GetMap: 

\url{http://mapy.kr-stredocesky.cz/ids_zony_wms?SERVICE=WMS&REQUEST=GetMap&VERSION=1.3.0&FORMAT=image/png&LAYERS=sid_zony&CRS=EPSG:2065&BBOX=-834258.9702,-1129697.611,-651189.0329,-968639.3882&WIDTH=1000&HEIGHT=1000}
%% \begin{alltt}\footnotesize
%% \href{\StredoceskygetMap}{http://mapy.kr-stredocesky.cz/ids\_zony\_wms?}
%% \href{\StredoceskygetMap}{SERVICE=WMS\&REQUEST=GetMap\&VERSION=1.3.0\&}
%% \href{\StredoceskygetMap}{&LAYERS=sid_zony\&FORMAT=image/png&CRS=EPSG:2065\&}
%% \href{\StredoceskygetMap}{BBOX=48.093144621684,11.6163532829661,51.4980192528993,19.0628256634265\&}
%% \href{\StredoceskygetMap}{WIDTH=1000\&HEIGHT=1000}
%% \end{alltt}

A výsledný dotaz GetFeatureInfo vypadá takto:
\newcommand{\StredoceskyGetFeatureInfo}{http://mapy.kr-stredocesky.cz/ids_zony_wms?REQUEST=GetFeatureInfo&VERSION=1.3.0&FORMAT=image/png&LAYERS=sid_zony&CRS=EPSG:2065&BBOX=-834258.9702,-1129697.611,-651189.0329,-968639.3882&WIDTH=1000&HEIGHT=1000&INFO_FORMAT=text/html&I=695&J=720&QUERY_LAYERS=sid_zony}
\begin{alltt}\footnotesize
\href{\StredoceskyGetFeatureInfo}{http://mapy.kr-stredocesky.cz/ids\_zony\_wms?}
\href{\StredoceskyGetFeatureInfo}{REQUEST=GetFeatureInfo\&VERSION=1.3.0\&}
\href{\StredoceskyGetFeatureInfo}{FORMAT=image/png\&LAYERS=sid\_zony\&CRS=EPSG:2065\&}
\href{\StredoceskyGetFeatureInfo}{BBOX=-834258.9702,-1129697.611,-651189.0329,-968639.3882\&WIDTH=1000\&HEIGHT=1000\&}
\href{\StredoceskyGetFeatureInfo}{INFO\_FORMAT=text/html\&I=695\&J=720\&QUERY\_LAYERS=sid\_zony}
\end{alltt}


Jak lze vidět, tento dotaz zarnuje dotaz GetMap a další parametry:
\begin{itemize}
  \item Parametr REQUEST uvádí typ dotazu GetFeatureInfo 
  \item INFO\_FORMAT tento parametr uvádí formát odpovědi na tento dotaz. WMS standatd neuvádí implicitní formát, který musí server podporovat. V tomto případě se jedná o html soubor. 
  \item Parametry I a J lokalizují prvek, na který se v dotazu GetFeatureInfo ptáme. Tyto souřadnice reprezentují souřadnicový systém obrázku s jednotkami v pixelech a začátkem v levém horním rohu. Souřadnice os I, J narůstají  
        směrem vpravo resp. dolů. Interval souřadnic je dán parametry WIDTH a HEIGHT, kdy I, J mají rozsah od 0 do WIDTH - 1 resp. od 0 do HEIGHT - 1.  
  \item QUERY\_LAYERS je výčet vrstev, kterých se týká dotaz GetFeatureInfo. 
\end{itemize}

  
\begin{figure}[h!]
 \includegraphics[scale=0.3]{figures/getfeatureinfo}
  \caption{Výsledek GetMap dotazu s bodem o souřadnicích I, J  695 a 720:}
\end{figure}

A jako odpověd obdržíme html dokunent, který se v prohlížeči zobrazí takto:

 \includegraphics[scale=0.7]{figures/getfeatureinforeply}


Pokud je serveru položen dotaz ve špatném tvaru, vrátí vyjímku implicitně ve formátu XML souboru, v němž je blíže specifikována chyba ve WMS dotazu. 


\newpage

\section{WMS modul pro GRASS}

Jelikož původně GRASS neměl žádné GUI, jeho moduly jsou přizpůsobeny pro práci v příkazové řádce. Každý modul si lze představit jako funkci, která má definované vstupní argumenty a určené výstupy. Nevýhodou tohoto 
přístupu je, že modul není schopen s uživatelem komunikovat za běhu. Uživatel může jeho chování ovlivnit jen pomocí argumentů, které jsou zadány před spuštěním.

Z tohoto důvodu implementace modulu odpovídá dotazu typu GetMap. Kdy uživatel musí zadat parametry tohoto dotazu a modul se postará a stažení rastru, jeho import do GRASS a případné další úpravy. 


Tento fakt ovšem nebrání úplné interaktivní implementaci podpory WMS do GRASS GIS. Je ji ovšem potřeba rozdělit do dvou částí. První částí je vytvoření WMS modulu, který umožní pracovat s WMS daty i uživatelům, jenž používají
pouze příkazovou řádku.  Do GUI programu je pak možné implementovat interaktivní část.


\subsection{Analýza původního stavu}

 V GRASS GIS již existuje modul r.in.wms umožňující získání WMS dat.
Velkým problémem tohoto modulu je rozdělení jeho kódu do 8 souborů, což jej činí velmi nepřehledným. 
Tento modul byl původně napsán jako shell skript. V GRASS 7 byly všechny moduly v shell skript  přepsány do jazyka Python, což také nepřispělo k jeho zpřehlednění. 

Ještě větším problémem však je, že tento modul selhává při komunikaci s mnoha WMS servery. Z vlasní zkušenosti musím konstatovat, že při používání modul více selhával, než pracoval správně.

Na základě těchto faktů bylo rozhodnuto, že se WMS modul pro GRASS vytvoří od základů znova. Případná oprava chyb ve stávajícím modulu a reorganizace kódu by byla mnohem problematičtější a časově náročnější.

\newpage

\subsection{Stanovení cílů implementace}

Návrh struktury modulu byl realizován na základě těchto požadavků:
\begin{itemize}
  \item Modul bude plně podporovat všechny možnosti dotazu GetMap.
  \item Modul bude komunikovat se serverem prostřednictvím knihovny GDAL\footnote{\url{www.gdal.org/}}  a také pomocí vlastní implementace 
  \item WMS severy mají nastaveny limity pro přenos dat, aby zabránily dotazům, které by je nadměrně zatížily. Tyto limity jsou dány formou maximálních hodnot parametrů WIDTH a HEIGHT dotazu GetMap. 
        Modul bude umět požadavek rozdělit na více WMS dotazů a získat tak požadovaný rastr po částech tzv. dlaždice, které posléze složí do jednoho rastru.
  \item Důležitým prvkem při práci v systému GRASS GIS je lokace. Lokace sdružuje data, která mají stejnou projekci. Její podmnožinou je mapset, který seskupuje data lokace do logických celků. 
        Na začátku práce v GRASS GIS si uživatel vybere lokaci ve které chce pracovat. Následná práce je svázána s touto lokací a s  jejím souřadnicovým systémem.
 
        Pokud by uživatel chtěl získat rastr z WMS serveru, který neposkytuje data v projekci lokace,  musel by nejprve vytvořit lokaci v projekci WMS dotazu a poté tyto data
         manuálně transformovat a zkopírovat do pracovní lokace.
	
	Proto bude modul schopen obdržená data automaticky transformovat  do souřadnicového systému lokace, pokud se jejich projekce bude lišit. 
  \item Jelikož existují další rozšíření standardu WMS, jako například Web Map Tile Service \footnote{\url{http://www.opengeospatial.org/standards/wmts}}, struktura modulu umožní snadnou implementaci 
        těchto rozšíření do existujícího kódu. 
  \item Vstupní argumenty modulu budou kompatibilní s modulem r.in.wms. Některé nevýznamné argumenty, které nemají vliv na funkčnost modulu, mohou být vynechány.
  \item Jako dplňkovou funkci bude modul schopen stáhnout a vypsat na standardní výstup obsah Capabilities souboru. Další zpracování toto výstupu bude součástí GUI.  
 \end{itemize}



\subsection{Volba způsobu implemntace}

V GRASS GIS verze 7 mohou být moduly implemntovány v jazyce C nebo Python. V jazyce C se implementují moduly, které jsou náročné na vypočetní výkon počítače. 
Rychlost je vykoupena značně delším vývojovým cyklem, protože programátor je nucen se starat o mnoho věcí, o které se Python postará sám.

Koncepce WMS modulu byla zvolena tak, aby neprováděl žádně složité výpočetní operace, ale aby pro tyto typy operací byly využity již implementované funkcionality 
ostatních GRASS modulů nebo knihoven. Proto byl vybrán jazyk Python. 

Pro práci modulu se staženým rastrem jako je reprojekce a spojení dlaždic byla zvolena knihovna GDAL, která je součástí standardní instalace GRASS.

Tato open source knihovna umožňuje čtění, zápis a reprojekci rastrů.  
Základním prvkem knihovny, který reprezentuje rastrová data je dataset.  Každý dataset reprezentuje rastrová data v určitém formátu, s nimiž pracuje pomocí driveru. Driver je třída, která je schopná číst a zapisovat data v určitém formátu. 

\subsection{Vstupní argumenty modulu}


Většina vstupních argumentů reprezentuje jednotlivé parametry dotazu GetMap. Počet řádků (HEIGHT), sloupců (WIDTH) a geografický rozsah dat (BBOX) 
jsou reprezentovány pomocí regionu. Region v GRASS je datová struktura, která definuje oblast na základě obdélníku. Tento obdélník je dán mezními kartografickými  souřadnicemi v každé ose a počtem řádků a sloupců pro rastr. Každý region je vztažen k určité projekci, která je totožná s projekcí lokace, ve které je uložen.

Pomocí regionu se určuje rozsah, na kterém bude aplikována činnost modulů.
Oblastí vně regionu nejsou do výpočtů zahrnuty. Vyjímkou jsou moduly, které data načítají jako třeba modul r.in.gdal. Tyto moduly implicitně načítají data v celém jejich rozsahu, bez ohledu na výpočetní region. 

Vstupní argumenty, které jsou nad rámec parametrů dotazu GetMap, se týkají dlaždicování. Modul umožnuje zadání maximálního počtu řádků 
a slopuců v jednom WMS dotazu. Na základě těchto hodnot se rozdělí získání dat z WMS serveru do několika dotazů a výsledný rastr je složen z rastů získaných těmito dotazy. 
   


\subsection{Implemntace modulu}

TODO UML Diagram

Třída WMSBase je abstraktní třída, která vykonává ty funkce, které se netýkají komunikace s WMS serverem. Tato komunikace je implementována v odvozených WMSGDALDrv a WMSDrv.

\subsubsection{Třída WMSBase}


Hlavní funkcí třídy WMSBase je výpočet parametrů pro komunikaci s WMS serverem, import již stažených dat do lokace a dodatečné úpravy těchto dat pomocí modulů GRASS. 

Třída obsahuje tyto důležité metody:  

\begin{itemize}
  \item \_GetCapabilities - Vytvoří a pošle WMS serveru dotaz typu GetCapabilities, následně obdrženou odpověď vypíše na standardní výstup a tímto se běh modulu ukončí. 
  \item \_GetMap - Tato metoda postupně volá níže zmíněné metody v pořadí v jakém jsou vedeny. 
  \item \_initializeParameters - Metoda inicializuje proměnné, potřebné pro další běh modulu. Jde o uložení hodnot ze slovníků vstupních argumentů do separátních proměnných a také získání hodnot ze zvoleného regionu.
  \item \_computeBbox - Pokud se liší projekce, ve které budou data staženy z WMS serveru a projekce pracovní lokace, je potřeba souřadnice, které vymezují region transformovat do projekce WMS dotazu, aby bylo možně definovat parametr BBOX ve správném souřadnicovém systému.
                        Zpravidla výsledkem transformace není obdélník se stranami rovnoběžnými se souřadnicovými osami systému, ale obecný čtyřúhelník. Protože parametr BBOX WMS dotazu GetMap  musí být uveden  v tomto obdélníku, jsou vybrány extrémní souřadnice, které tvoří obdélník rovnoběžný s osami souřadnicového systému.      
  \item \_download - Jedná se o metodu virtuální, definovanou v potomcích třídy, která dotazem GetMap stáhne rastr a uloží jej do dočasného souboru. Zpravidla jsou rastrová data poskytovaná WMS serverem tříkanálová reprezentující RGB systém barev. Pokud jde o rastr s průhlednými plochami, obsahuje ještě alfa kanál, který definuje průhlednost pixelů. Návratovou hodnotou metody je cesta k souboru s rastrem.
  \item \_createOutputMap - Pokud je to potřeba, tato metoda nejprve pomocí utility gdalwarp knihovny GDAL transformuje rastr do projekce lokace. Transformovaný rastr je uložen do nového souboru. 
                            Jelikož rastr je matice o určitých počtech sloupců a řádků, čemuž ale tvar transformovaného rastru zpravidla již neodpovídá, je potřeba mít informaci o tom, který pixel odpovídá původnímu rastru 
                            a který pixel již nereprezentuje data původního rastru. Tato informace je zahrnuta do alfa kanálu. Pixely, které do původnímu rastru nepatří mají hodnotu v alfa kanálu plně průhlného pixelu.  Pokud transformovaný rastr neobsahuje alfa kanál, utilita gdalwarp jej přidá.

Posléze pomocí modulu r.in.gdal importuje rastrová data do lokace.
Modul naimportuje rastr do vrstev po jednotlivých kanálech. Tyto vrstvy jsou pomocí modulu  r.composite sloučeny do jedné barevné vrstvy. 

Aby byla správně nastavena průhlednost výsledné rastrové vrstvy, před spuštěním modulu r.composite, se vytovří z alfa kanálu inverzní maska. Maska se v GRASS GIS používá v těch případech, kdy rozsah dat, nad nimiž bude modul pracovat, nelze vyjádřit pomocí regionu. Maska je v GRASS reprezentována rastrovou vrstvou, jejíž název je MASK. Pixely, kde je maska definována 
nejsou brány při výpočtu v úvahu. 

Při použití masky vytvoří modul r.composite barevný rastr, který správně zobrazí pixely rastru, které mají být průhledné.

\end{itemize}
Modul za svého běhu vyprodukuje několik dočasných souborů. Tyto dočasné soubory jsou smazány ihned poté, co již nejsou potřeba. Pokud však uživatel neočekávaně přeruší běh programu, může se stát, že některý soubor nebude smazán. Aby se tomuto zabránilo,
je v destruktoru třídy, který se volá i při neočekávaném ukončení modulu, provedeno smazání těchto souborů, pokud nebyly odstraněny. Toto je provedeno i s vrstvami, které reprezentují jednotlivé barevné kanály a maskou. 





\subsubsection{Třída WMSGDALDrv}

Tato třída přistupuje k datům WMS serveru prostřednictvím WMS driveru knihovy GDAL. Parametry WMS dotazu jsou driveru předány ve formě XML souboru \footnote{\url{http://www.gdal.org/frmt_wms.html}}. 
Driver následně stáhne data a uloží je do souboru ve formátu GeoTiff. 



\subsubsection{Třída WMSDrv}

Tato třída komunikuje se serverem přímo, bez použití další knihovny. Nejprve je ze vstupních parametrů vytvořen WMS dotaz. 
Pak následuje výpočet rozměrů dlaždic v souřadnicovém systému projekce, rozměru dlaždic v pixelech a jejich počet. Na základě těchto hodnot jsou v cyklu staženy jednotlivé dlaždice, kdy je k WMS dotazu přidán parametr BBOX pro konkrétní dlaždici, tento dotaz je poslán WMS serveru a do dočasného
souboru je uložen stažený rastr. Spojování dlaždic do jednoho rastru je řešeno pomocí knihovny GDAL. Při prvním průchodu cyklu se vytvoří nový dataset, kde se dlaždice postupně, tak jak jsou stahovány, spojují. 

\subsection{Problémy a jejich řešení}

Při implementaci třídy WMSDrv se vyskytl problém při dlaždicování.
Při stažení barevného jednokanálového rastru (např. PNG, GIF) s přiloženou globální tabulkou barev vznikl po spojení jednotlivých dlaždic rastr, který tuto tabulku neobsahoval.

Protože v tomto případě konkrétní barvy jednotlivých kanálů jsou definovány až v tabulce barev, není možné bez této tabulky správně interpretovat hodnoty pixelů, které  zde představují pouze odkazy na položky tabulky. V různých dlaždicích jsou stejné barvy definovány jinými položkami tabulky barev a hodnota pixelů na ně odkazující se v nich liší. 
Při spojování dlaždic knihovna GDAL nebrala tabulku barev v úvahu a interpretovala přímo hodnoty těchto pixelů. Výsledný rastr se spojenými dlaždicemi vypadal například takto: 

 \includegraphics[scale=0.4]{figures/color_table_problem.png}



Tento problém byl vyřešen pomocí upraveného kódu GDAL utility pct2rgb \footnote{\url{www.gdal.org/pct2rgb.html}}. Tento kód obsahuje metoda \_pct2rgb třídy  WMSGDALDrv.
Tato metoda vytvoří 4 kanálový dataset (RGB + alfa vrstva), a do jednotlivých kanálů jsou pixel po pixelu přiřazovány jejich hodnoty z tabulky barev. 

\subsection{Ukázka práce modulu}

\subsection{Budoucí vývoj modulu}

\newpage

\section{ WMS modul pro SAGA GIS}

V SAGA GIS je koncept modulu odlišný od GRASS GIS. Jedním z důvodů tohoto rozdílu může být to, že SAGA je software mnohem mladší než GRASS, proto v době, kdy byl jeho vývoj započat, bylo grafické rozhraní programů již standardem, narozdíl 
 od počátku vývoje GRASS GIS. 

Proto moduly SAGA GIS mohou být více interaktivní a za běhu komunikovat s uživatelem, čehož lze dobře využít při vývoji WMS modulu.

\subsection{Analýza původního stavu}

Do aktuální verze (2.0.8) SAGA GIS byla nově začleněna experimentální knihovna Garden - Web Service Data Access. Tato knihovna obsahuje modul Import a Map via Web Map Service (WMS), který umožňuje získat data z WMS serveru.
Modul funguje tak, že po spuštění vytvoří ze zadané URL adresy dotaz GetCapabilities, na základě odpovědi vytvoří dialog, ve kterém uživatel může vybrat parametry pro získání rastru z WMS serveru. 

Tento koncept modulu je uživatelsky velmi přívětivý, jelikož stačí znát pouze URL adresu a o vše ostatní se postará modul. Uživatel si pak vybere z možností, které server nabízí a nemusí tyto informace zjišťovat jiným způsobem. 


Jelikož se jedná o experimantální modul, obsahuje velké množství nedostatků. Zejména bych chtěl uvést tyto:

\begin{itemize}
\item Nesprávně zpracovává Capabilities soubor, a proto vygeneruje dialog, který nedovoluje uživateli vybrat si ze všech možností, které server nabízí. Omezení dialogu jsou tato:
\begin{itemize}
 \item     V dialogu nenabídne všechny vrstvy, ale pouze ty, které jsou přímými potomky kořenové vrstvy, což je výrazné umezení, jelikož se týká přímo poskytovaných dat WMS serveru.
 \item 	   Nelze určit pořadí vrstev, v jakém budou  zobrazeny na výsledné mapě.
 \item 	   Neumožňuje výběr stylu vrstvy .
 \item 	   Při načítání vrstev nebere v úvahu dědičnost	  . 
\end{itemize}
\item Nepodporuje dlaždicování. 
\end{itemize}


 
\subsection{Stanovení cílů implementace}


Základní koncept modulu bude stejný jako u modulu experimentálního. To znamená, že modul nejprve získá Capabilites soubor z WMS serveru, vygeneruje pro něj nabídku a na základě zvolených hodnot pomocí 
dotazu GetMap stáhne požadovaný mapový rastr a naimportuje jej do programu. 

Při implementaci by měly být naplněny tyto cíle:
\begin{itemize}
\item  Modul bude schopen správně schopen zpracovat Capabilities soubor ve standardu 1.1.1 a 1.3.0.  Jelikož ne všechny WMS servery plně podporují WMS standardy, bude napsán tak, aby si poradil s těmi prohřežky proti standardu, které jsou nejčastější, jako je například neuvedení informace o stylech vrstvy.  
\item Nabídka parametrů umožní uživateli využití všech možností WMS dotazu GetMap.
\item Modul bude schopen transformovat rastr do uživatelem požadováného zobrazení. V SAGA GIS neexistuje ekvivalent lokace jako v GRASS GIS. Z vrstvami lze pracovat v libovolném zobrazení najednou. Existuje i modul,
      který umí trasformovat rastr do jiné projekce. Tato funkcionalita ulehčí uživateli práci, protože často se stává že WMS server neposkytuje data v uživatelem požadované projekci. 
Také modul umožní zadat hodnotu parametru BBOX v souřadnicích požadované projekce a za běhu modulu, tyto souřadnice transformovat do projekce WMS dotazu.
\item Umožní implementovat další rozšíření WMS standardu. 
\end{itemize}


\subsection{Volba způsubu implemntace}

Modul byl implementován v jazyce C++, ve kterém je napsán celý program SAGA a všechny jeho moduly.  Byla snaha využít objektových rysů jazyka C++, které mohou ulehčit budoucí rozšířování 
modulu o podporu dalších nádstaveb WMS a celkově pomoci vytvořit čitelnější kód. 

Další důležitou volbou, před vlastní implementací modulu, bylo rozhodnutí, zda v tomto modulu spouštět další SAGA moduly, které  by například transformovaly rastr do požadované projekce nebo transformormovaly souřadnice. 

Nakonec bylo rozhodnuto tyto operace provést před importem do SAGA GIS pomocí knihovny GDAL (reprojekce) a knihovny PROJ4 (transformace souřadnic), jelikož spouštění ostatních modulů z modulu není v SAGA GIS tak jednoduché jako v GRASS, kde je možné modul zavolat prostřednictvím jediné funkce. 

\subsubsection{Knihovna PROJ4}

Knihovna PROJ4 \url{http://trac.osgeo.org/proj/} je open source projekt, který dovoluje transformovat souřadnice do různých projecí, které jsou součástí této knihovny.Také umožňuje definovat vlastní 
projekce. Tuto knihovnu používá např. GRASS GIS, SAGA GIS nebo GDAL.

\subsection{Moduly v SAGA GIS}

Každý modul v SAGA GIS je potomkem třídy CSG\_Modul. Vstupní argumenty modulu, které se zadávají před spuštěním jsou uvedeny v konstruktoru třídy modulu.

Argumenty jsou prvky instance třídy CSG\_Parameters Parameters. 
Metoda OnExecute je další metoda, kterou musí každý modul obsahovat. Tato metoda je volána při spuštění modulu. 

V SAGA GIS existují ještě interaktivní moduly, které jsou schopny reagovat např. na kliky v mapovém okně nebo na stisky kláves a jejich struktura se liší.    


\subsection{Implemntace modulu}
Modul je implementován jako přímý potomek třídy CSG\_Module. Tato třída se nazývá CWMS\_Import a má na starosti všechny funkce, které jsou přímo spojeny s programem SAGA GIS.

Jedná se o vytvoření dialogů na základě Capabilities souboru, který je stažen a načten  skrz členy třídy CWMS\_Base a o import rastru do programu. 

\subsubsection{Třída CWMS\_Base}

Tato třída nejprve pomocí instance CWMS\_Capabilities, získá Capabilities data z WMS serveru. Na základě těchto dat třída CWMS\_Import vygeneruje dvě nabídky, ze kterých uživatel vybere parametry dotazu GetMap. 

Potom je spuštěna metoda GetMap, která stáhne a transfomruje rastr do požadované projekce. Tado metoda vrátí cestu k souboru, ve kterém je uložena již výsledná rastrová mapa určená k načtení do programu.
Struktura a funkce této metody jsou velmi podobné WMS modulu pro GRASS. Největším rozdílem je to, že nenačítá výslednou mapu do programu, protože k tomu je potřeba využít neveřejných členů třídy CSG\_Module, 
které nejsou z této třídy dostupné. Toto by šlo obejít deklarováním CWMS\_Base přítelem CWMS\_Import, což je potomek CSG\_Module. Tato deklarace však porušuje zapouzdřenost, jeden za základních principů objektově orientovaného
programování. Proto je metoda \_Import\_Map součástí třídy CWMS\_Import, z níž může přistupovat k těmto členům.

Metoda GetMap volá tři neveřejné metody v tomto pořadí:
\begin{itemize}
 \item \_ComputeBbox - Transformuje souřadnice obdélníku, pokud jsou zadány v jiné projekci než, kterou nabízí WMS server. Transformace je provedena pomocí knihovny PROJ4, která je standardně přítomna v každé SAGA instalaci. Z transformovaných bodů jsou vybrány extrémní souřadnice, které pak definují parametr BBOX WNS dotazu. 
 \item \_Download - Tato metoda je virtuální a je pouze v této třídě deklarovaná. 
 Návratovou hodnotou této metody je rastr, ve kterém jsou již spojeny dlaždice
 z jednotlivých WMS dotazů.
 \item \_GdalWarp - Tato metoda pomocí knihovny GDAL, pokud je to potřeba, transformuje stažený rastr, do výsledné projekce. Jedná se o modifikovaný kód z \url{http://www.gdal.org/warptut.html}.
\end{itemize}

\subsubsection{Třída CWMS\_Gdal\_drv}

Členění této třídy a funkce jejich metod jsou v zásadě stejné jako u třídy WMSGDALDrv v GRASS modulu. Metoda \_CreateGdalDrvXml vytovří XML soubor s parametry pro GDAL WMS driver, 
který metoda  \_Download použije pro stažení rastru z WMS serveru. 


\subsubsection{Třída CWMS\_Capabilities}

Tato  třída pomocí metody Create, jejíž argumentem je URL adresa WMS serveru, vytvoří WMS dotaz GetCapabilities a následně načte do své struktury XML soubor s odpovědi.

K načtení XML souboru jsou použity třídy knihovny WxWidgets, která je součástí programu SAGA. Pomocí objektu wxXmlDocument této knihovny je načten XML soubor a vytvořena datová struktura tohoto souboru. Ekvivalentem XML elementu v této struktuře je třída wxXmlNode, která obsahuje odkaz na rodiče a na své přímé potomky. 

Pokud by neexistovala dědičnost ve stromu elementů <Layer>, bylo by možné snadno zjistit všechny informace o vrstvě z jejího wxXmlNode a stačilo by pouze uchovat ve třídě CWMS\_Capabilities odkaz na kořenový element XML souboru. Protože však tato dědičnost existuje, bylo potřeba nějakým způsobem ke každé vrstvě seskupit všechny její elementy, včetně zděděných. 
  

Proto byla vytvořena třída  CWMS\_Layer, která obsahuje ukazatel na rodičovskou vrstvu a na její přímé potomky ve stromu. Dalším členem třídy je kontejner map s názvem m\_LayerElms, který obsahuje položky, mající jako klíč název elementu, který je přímým potomkem elemetnu <Layer> a vektor ukazatelů na objekty wxXmlNode reprezentující tyto elementy. Tento kontejner zahrnuje všechny elementy, včetně elementů získaných dědičností. Jelikož se jedná o ukazatele na objekty, nejsou ukládány duplicitní informace, ale všechny vrstvy, které dědí stejný element odkazují na tentýž objekt. 

Toto uspořádání je vytvořeno rekurzivně, kdy je strom elementů <Layer> procházen od kořene k jeho listům. Po vytvoření instance  CWMS\_Layer je zavolána její metoda Create, která z kontejneru m\_LayerElms  rodičovského objektu třídy, který je již díky rekurzi vytvořen, vybere ty objekty, které dědí a naplní vlastní kontejner m\_LayerElms. Aby kontejner obsahoval všechny přímé potomky elementu <Layer>, jsou potom přidány elementy, které se nedědí. 

\subsection{Ukázka práce modulu}

\subsection{Budoucí vývoj modulu}

\end{document}







